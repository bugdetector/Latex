\chapter{General Information}
Istanbul is a beautiful city of stunning architecture, history and culture. You'll find ancient and modern colleges, fascinating museums and galleries, and plenty of parks, gardens and green spaces in which to relax. Although the city is spread over a large area, you will have easy reach to anywhere you would like to go thanks to a variety of modern and developed transportation systems as well as interchangeable rail systems to long-way metrobus lines.

\begin{figure}[!htbp]
\centering
\includegraphics[width=\textwidth]{projectChapters/images/Picture1.png}
\caption{Landscape design of Yıldız Technical University}
\label{fig:ornek1}
\end{figure}

In Figure-\ref{fig:ornek1}, landscape design of Yıldız Technical University is illustrated. The side view of the garden can be seen from Figure \ref{fig:ornek1}.

\section{History}
The stages our university has passed through\cite{WinNT} in its distinguished past are outlined below. Kondüktör Mekteb-i Âlisi/ The Conductors (Technicians) School of Higher Education (1911-1922). The Kondüktör Mekteb-i Âlisi/Conductors (Technicians) School of Higher Education was founded in 1911 in order to meet the “science officer” (known previously as conductors, and today as technicians) requirement of the Municipality Public Works Section. The school was modeled on the syllabus of the “Ecole de Conducteur” and was affiliated with the Ministry of Public Works. Enrolment began on 22 August 1911.


\textbf{Nafia Fen Mektebi/ The School of Public Works (1922-1937):}  The school’s name was changed to Nafia Fen Mektebi/ School of Public Works in 1922 and the duration of education was increased to 2.5 years in 1926 and 3 years in 1931.

\section{Historical Advancements in the University}

The school was established as an autonomous higher education and research institution with Law no. 1184 of State Engineering and Architectural Academies published on 3 June 1969. 

Law no. 1472 ruled for the closing of special vocational schools in 1971, and engineering schools were affiliated with the Istanbul State Engineering and Architectural Academy.

\subsection{The Yıldız University Period}
The Istanbul State Engineering and Architectural Academy and affiliated schools of engineering and the related faculties and departments of the Kocaeli State Engineering and Architecture Academy and the Kocaeli Vocational School were merged to form Yıldız University with decree law no.41 dated 20 June 1982 and Law no. 2809 dated 30 March 1982 which accepted the decree law with changes.

The new university incorporated the departments of Science-Literature and Engineering, the Vocational School in Kocaeli, a Science Institute, a Social Sciences Institute and the Foreign Languages, Atatürk Principles and the History of Revolution, Turkish Language, Physical Education and Fine Arts departments affiliated with the Rectorate.

\subsection{The Yıldız Technical University Period}
Law no. 3837 dated 3 July 1992 renamed our university Yıldız Technical University. The Engineering Faculty was divided into four faculties and restructured as the Electrical-Electronics, Construction, Mechanical and Chemical-Metallurgy Faculties and also included the Faculty of Economics and Administrative Sciences within its organization. The Kocaeli Faculty of Engineering and the Kocaeli Vocational School were released from our university to be restructured as Kocaeli University. Today our university has 9 Faculties \footnote{2 of these faculties are located at Yildiz Campus, and the other ones are located at Davutpaşa Campus}, 2 Institutes, the Vocational School of Higher Education, the Vocational School for National Palaces and Historical Buildings, the Vocational School for Foreign Languages and more than 20,000 students. 

\begin{figure}[htbp]
\centering
\includegraphics[width=\textwidth]{projectChapters/images/Picture3}
\caption{Side view of Graduate School of Natural and Applied Sciences, Yıldız Technical University, Çukursaray, İstanbul}
\label{fig:ornek2}
\end{figure}

\subsubsection{Mission}
Our mission is to create a university which pioneers education, scientific research, technological development and artistic work aimed at the progress of society and the increase of the quality of life within an understanding of national and international solidarity; and educates creative, enterprising, questioning and ethical students equipped with universal values, who constantly renew themselves, aim for lifelong learning and are capable of analysis and synthesis.

\begin{equation}
\Delta  l = h \Delta \theta
\end{equation}
\begin{equation}
\frac{\Delta l}{\Delta t} = h \frac{\Delta \theta}{\Delta t}
\end{equation}
\begin{equation}
\lim_{\Delta t \rightarrow 0}\frac{\Delta l}{\Delta t} = \lim_{\Delta t \rightarrow 0} h \frac{\Delta \theta}{\Delta t}
\end{equation}
\begin{equation}
\lim_{\Delta t \rightarrow 0}\frac{\Delta l}{\Delta t} = \frac{dI}{dt}
\end{equation}