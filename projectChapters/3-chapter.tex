\chapter{Feasibility}
This system needs requirements listed below:
\begin{itemize}
    \item This system needs real time location of user. So this system needs to run on mobile devices.
    \item This system must provide interaction within users. So to provide intraction of multiple mobile devices this system needs a server side application.
    \item In process of development to provide version controlling Git must used as VCS.
\end{itemize}

\section{Technic Feasibility}
As technical feasibility study, the software and hardware needs for the project is defined on the following sections.
\subsection{Software Feasibility}

In this project a web and a mobile application will be developed. The following software
technologies will be used.

\textbf{Mobile Side Development}
\newline
The tools and development environments used for mobile side development of the
project are mentioned below.
\newline
• Android: Android is a mobile operating system based on the Linux kernel. Its
source code is licensed under open source licenses and it is developing by Google
and Open Handset Alliance. The top layer of Android’s architecture is called
The Application Framework layer and it provides many higher-level services to
applications in the form of Java classes. Android was chosen over iOS because
of publishing problems, restrictions and lack of design guidelines that come with
iOS\cite{android}.
\newline
• Android Studio: Android Studio is the official IDE for Android application
development, based on IntelliJ IDEA. Android Studio offers some advantages
over Eclipse, such as Gradle based flexible build system, advanced layout editor,
built-in Git source control and Maven library support\cite{androidStudio}.
\newline

• Android Sqlite Database: SQLite is a opensource SQL database that stores data to a text file on a device. Android comes in with built in SQLite database implementation. SQLite supports all the relational database features\cite{androidStudioSqlite}.
\newline

• Android Emulator: Android emulator lets prototype, develop and test Android applications without using a physical device\cite{androidEmulator}.
\newline

• Google Map API: Google Maps APIs give developers several ways of embedding Google Maps into web pages or retrieving data from Google Maps, and allow for either simple use or extensive customization\cite{googleMapAPI}.
\newline

\textbf{Server Side Development}

The tools and development environments used for server side development of the
project are mentioned below.
\newline
• Java EE:

• Eclipse:

• Sublime Text:

• MySQL Workbench:

• STS:

• Google Map API:


\subsection{Hardware Feasibility}
The minimum and recommended hardware requirements for each program/IDE and
a system requirement compilation for development is shown in Table 3.1 based on
the requirements.

For this project we selected to rent a cloud computing system for make this project scalable. Scaleway\cite{scaleway} provides cloud computing services. Scaleway can provide multiple datacenters on different locations and developer tools on the machines. We selected scaleway for renting cloud server(s). Starter package is enough for this project at startup.

\begin{table}[!ht]
\centering
\caption{Scaleway package options}
\label{minreq}
\begin{tabular}{|c|c|c|c|c|}
\hline
\textbf{Specification}& \textbf{Starter} & \textbf{C2}  & \textbf{ARM64} & \textbf{C1} \\ \hline
CPU                             & 2x86 64bit & 8x86 64bit  & 8xARMv8 & 4xARMv7 \\ \hline
RAM                             & 2GB & 32GB & 8GB & 2GB \\ \hline
Storage                         & 50GB SSD & 50GB SSD & 200GB SSD & 50GB SSD \\ \hline
Number of public IPv4  & 1 & 1 & 1 & 1  \\ \hline
Bandwidth                       & 200Mbit/s & 800Mbit/s & 200Mbit/s & 200Mbit/s \\ \hline
\end{tabular}
\end{table}

\begin{table}[!ht]
\centering
\caption{Minimum System Requirement for Mobile Application Development}
\label{minreq}
\begin{tabular}{|c|c|c|c|}
\hline
\textbf{Software}& \textbf{CPU} & \textbf{RAM}  & \textbf{Storage} \\ \hline
Linux OS\cite{linuxMinimumSystemRequirements} & 1GHz & 512 MB & 8 GB \\ \hline
Android Studio\cite{androidMinimumSystemRequirements} & 1.6GHz & 3 GB RAM  & 8GB(500MB for IDE, 7.5 GB for SDK) \\ \hline
Android Emulator\cite{androidMinimumSystemRequirements} & unknown & 1GB & 1.5 GB \\ \hline
\end{tabular}
\end{table}

\begin{table}[!ht]
\centering
\caption{Minimum System Requirement for Server Side Application Development}
\label{minreq}
\begin{tabular}{|c|c|c|c|}
\hline
\textbf{Software}& \textbf{CPU} & \textbf{RAM}  & \textbf{Storage} \\ \hline
Windows OS\cite{microsoft} & 1GHz & 2 MB & 20 GB \\ \hline
Eclipse\cite{androidMinimumSystemRequirements} & 1.5GHz & 1 GB RAM  & 1 GB  \\ \hline
MySQL\cite{mysql} & 2 core & 2GB & 800 MB \\ \hline
MySQL Workbench\cite{mysql} & unknown & 4GB & 200 MB \\ \hline
\end{tabular}
\end{table}

\begin{table}[!ht]
\centering
\caption{Android Hardware Requirements}
\label{androidhardware}
\begin{tabular}{|c|c|c|c|}
\hline
            & \textbf{Minimum}    & \textbf{Recommended} \\\hline
CPU & 1 Ghz      & 2 Ghz     \\\hline
RAM       & 512 MB       & 2 GB        \\\hline
Storage   & 2 GB       & 8 GB       \\\hline
\end{tabular}
\end{table}

\subsection{Communication Feasibility}

Media files and path files are seperately stored on users android device. It so hard to send all of trip data one by one to one device to another without any loss. So we thought about that to compressing all files in a zip file before sending from mobile device to web server or vice versa. So we sended all data in one file. Also we will use less data bandwidth with compress all files in a zip file.
The Internet is the main communication technology used in this project. The
anticipated communication variables are shown in Table \ref{table:commparameters}.

\begin{table}[!ht]
\centering
\caption{Communication parameters}
\label{table:commparameters}
\begin{tabular}{|l|l|r|r|}
\hline
\textbf{Description}                                                                   & \multicolumn{1}{l|}{\textbf{Symbol}} & \multicolumn{1}{l|}{\textbf{Values}} \\ \hline
Average Path Size & A  & 50 kB \\ \hline
Average Video Size & B & 10 MB \\ \hline
\begin{tabular}[c]{@{}l@{}}Average Video Number\\Per User\end{tabular} & C & 4 \\ \hline
Average Photo Size & D & 2 MB \\ \hline
\begin{tabular}[c]{@{}l@{}}Average Photo Number\\Per User\end{tabular} & E & 20 \\ \hline
Average Sound Record Size & F & 1 MB \\ \hline
\begin{tabular}[c]{@{}l@{}}Average Sound Record\\Number Per User\end{tabular} & G & 1 \\ \hline
\begin{tabular}[c]{@{}l@{}}Average Number Of\\Members in Trip\end{tabular} & H & 2 \\ \hline
\begin{tabular}[c]{@{}l@{}}Average Trip Data\\Size on Mobile(Unzipped)\end{tabular}  & I & 81 MB \\ \hline
\begin{tabular}[c]{@{}l@{}}Average Trip Data\\Size on Web(Unzipped)\end{tabular}  & J & 162 MB \\ \hline
\begin{tabular}[c]{@{}l@{}}Average Trip Data\\Size on Mobile(Zipped)\end{tabular}  & K & 113 MB \\ \hline
\begin{tabular}[c]{@{}l@{}}Average Number of\\Upload Rate\end{tabular}  & L & 25/month \\ \hline
\begin{tabular}[c]{@{}l@{}}Average Number of\\Download Rate\end{tabular}  & M & 50/month \\ \hline
\begin{tabular}[c]{@{}l@{}}Average Size of\\Uploaded Trips\end{tabular}  & N & 2 GB \\ \hline
\begin{tabular}[c]{@{}l@{}}Average Size of\\Downloaded Trips\end{tabular}  & O & 8 GB \\ \hline
\begin{tabular}[c]{@{}l@{}}Supposed Number\\of Users\end{tabular}  & P & 100 \\ \hline
\begin{tabular}[c]{@{}l@{}}Average Zip\\Compression Ratio\end{tabular}  & R & 30\%\cite{zip} \\ \hline
\begin{tabular}[c]{@{}l@{}}Server Data Rate\\Per Month\end{tabular}  & S & 10 GB \\ \hline
\end{tabular}
\end{table}


I = G * F + E * D + C * B + A

J = I * H

K = I * (1-R)

N = I * L

O = J * M

S = O + N

Based on the calculations above, the monthly data size will be around 10 GB. Scaleway provide 2 GB ram and 50 GB storage with expandable options. As a result these properties are sufficent.

\section{Labor Force Feasibility}
There are two people needed for developing mobile and server side of system concurrently.



\section{Time Feasibility}
Gannt diagram shown below.
\begin{figure}[!htbp]
\centering
\includegraphics[width=\textwidth]{projectChapters/images/gantt.png}
\caption{Gantt Diyagramı Zaman Çizelgesi}
\end{figure}


\section{Legitimate Feasibility}
Software which is used within the project does not face any legal issues. All of the
software used in the project contain license requirements. Users are responsible for
all shared content. The users are responsible of their well being. So any misusing of 
any sharing content is their risk. So sharing and publishing content is on their own risk.


\section{Economic Feasibility}
The salary determined by the government of the Republic of Turkey for engineers
is 3.500 TL\cite{muhendisMaas}. The amount of assumed work days in a month is 21 days. The
individual and general cost of the personnel and features
the cost of hardware and software used for development are shown below.

\begin{table}[!h!]
\centering
\caption{Personnel cost table}
\label{tab:maas}
\begin{tabular}{|l|r|r|r|}
\hline
& \multicolumn{1}{l|}{\textbf{Time(Day)}} & \multicolumn{1}{l|}{\textbf{Price(TL/Day)}} & \multicolumn{1}{l|}{\textbf{Total(TL)}} \\ \hline
System Programmer   & 37                                      & 170 TL                                      & 6.290 TL                                 \\ \hline
System Analyst      & 29                                      & 170 TL                                      & 4.930 TL                                 \\ \hline
System Designer     & 10                                      & 170 TL                                      & 1.700 TL                                 \\ \hline
Project Coordinator & 4                                       & 170 TL                                      & 680,TL                                  \\ \hline
General Total:      & \multicolumn{3}{r|}{13.600 TL}                                                                                                  \\ \hline
\end{tabular}
\end{table}

\begin{table}[!h!]
\centering
\caption{Hardware and software used for development cost table}
\label{tab:hardsoftcost}
\begin{tabular}{|l|r|}
\hline
\textbf{Components}             & \multicolumn{1}{l|}{\textbf{Price}} \\ \hline
Eclipse \cite{eclipse} & Free \\ \hline
Android Studio \cite{androidStudio} & Free \\ \hline
2* Dell Vostro 5468 & 662 TL \\ \hline
General Mobile GM6 \cite{gm6} & 375 TL                            \\ \hline
2* Github Account                  & Free                             \\ \hline
Scaleway Server                  & 15TL/monthly                             \\ \hline
General Total:                  & 1.112 TL                            \\ \hline
\end{tabular}
\end{table}
The cost of computers used in the development process in 3181 TL\cite{dell}. In the process 2 computer were used. A computer can be used for 2 years so 48 months in average lifetime. From here, we will find the cost of computer as 662TL. General Mobile GM6 used in the development process is 900 TL\cite{gm6}. It can be used for 12 months in average lifetime. From here we will find the cost of phone as 375 TL. Considering these expenditures, the cost of development of our project is X TL.


\begin{table}[]
\centering
\caption{Total Cost Table For Gezi-Yorum}
\label{my-label}
\begin{tabular}{|l|l|}
\hline
\multicolumn{1}{|c|}{\textbf{Cost}} & \multicolumn{1}{c|}{\textbf{TL}} \\ \hline
Hardware                               & 5.000,00 TL                      \\ \hline
Project Team                           & 16.000,00 TL                      \\ \hline
Google Map API                         & 15,00 TL                      \\ \hline
AWS                                 & 60,00 TL                      \\ \hline
\textbf{Toplam}                        & \textbf{21.075,00 TL}            \\ \hline
\end{tabular}
\end{table}




