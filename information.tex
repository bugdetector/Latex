%%%%%%%%%%%%%%%%%%%%%%%%%%%%%%%%%%%%%%%%%%%%%%%%%%%%%%%%%%%%%%%%%%%%%%%
%%%%%%%%% Aşağıda istenilen bilgileri dikkatlice doldurunuz.   %%%%%%%%
%%%%%%%%% Doldurmanız istenilen ifadenin sonunda TR ya da EN   %%%%%%%%
%%%%%%%%% yazıyorsa, sırasıyla Türkçe veya İngilizce olarak    %%%%%%%%
%%%%%%%%% doldurunuz. Eğer herhangi bir ifade yoksa, projenizi %%%%%%%%
%%%%%%%%% hangi dilde yazıyorsanız (Türkçe veya İngilizce), o  %%%%%%%%
%%%%%%%%% dile göre doldurunuz. İsimleri yazarken soyisimleri  %%%%%%%%
%%%%%%%%% büyük harf ile yazınız.                              %%%%%%%%
%%%%%%%%%%%%%%%%%%%%%%%%%%%%%%%%%%%%%%%%%%%%%%%%%%%%%%%%%%%%%%%%%%%%%%%
%%%%%%%%%%%%%%%%%%%%%%%%%%%%%%%%%%%%%%%%%%%%%%%%%%%%%%%%%%%%%%%%%%%%%%%


% Proje başlığını Türkçe olarak yazınız.
\def\titleTR{ }

% Proje başlığını İngilizce olarak yazınız.
\def\titleEN{ Gezi-Yorum }

% Proje grubundaki ilk ismi yazınız.
\def\studenti{Tarık Nural}
%İlk öğrencinin, öğrenci numarasını yazınız.
\def\numberi{13011036}
% Proje grubundaki ilk öğrencinin doğum tarihi ve yerini yazınız.
\def\studentibdate{07.01.1995, İstanbul}
% Proje grubundaki ilk öğrencinin e-mail adresini yazınız.
\def\studentiemail{tariknural00@gmail.com}
% Proje grubundaki ilk öğrencinin cep telefonu numarasını yazınız.
\def\studentiphone{ 0535 261 1429}
% Proje grubundaki ilk öğrencinin staj deneyimlerini yazınız. Satır atlatmak için \\ kullanabilirsiniz.
\def\studentiintern{ SSI SCHAEFER \\ ARÇELİK A.Ş}

% Proje grubundaki ikinci ismi yazınız. Eğer ikinci üye yoksa ~ işareti ekleyiniz ve ikinci
% öğrenci ile alakalı diğer bilgileri atlayınız.
\def\studentii{Murat Baki Yücel}
%İkinci öğrencinin, öğrenci numarasını yazınız.
\def\numberii{13011035}
% Proje grubundaki ikinci öğrencinin doğum tarihi ve yerini yazınız.
\def\studentiibdate{04.01.1996, Kayseri}
% Proje grubundaki ikinci öğrencinin e-mail adresini yazınız.
\def\studentiiemail{bakiyucel38@gmail.com}
% Proje grubundaki ikinci öğrencinin cep telefonu numarasını yazınız.
\def\studentiiphone{ 0507 915 8686}
% Proje grubundaki ikinci öğrencinin staj deneyimlerini yazınız. Satır atlatmak için \\ kullanabilirsiniz.
\def\studentiiintern{ Pronic Yazılım \\ Kartaca Bilişim}

% Projeyi teslim ettiğiniz ay ve yılı proje için kullandığınız dilde yazınız.
\def\date{January, 2018}

% Proje danışmanınızın ismini Türkçe ünvanı ile yazınız.
\def\advisorTR{Yrd. Doç. Dr. Ahmet Tevfik İNAN}
% Proje danışmanınızın ismini İngilizce ünvanı ile yazınız.
\def\advisorEN{Assist. Prof. Dr. Ahmet Tevfik İNAN}

\def\acknowledgementText{
    % Buraya teşekkür metninizi proje için kullandığınız dilde yazınız. 
    We would like to express our sincere gratitude to our advisor "Assist. Prof. Dr. Ahmet Tevfik INAN" for the continuous support to our graduation project, his patience, motivation, and knowledge.
}

\def\abstractTextEnglish{
    % Buraya İngilizce olarak proje özetini yazınız.
The goal of this project is to record travel routes of people's trips using mobile device location and to add and edit media on the route. The developed system can be used to organize personal or team trips and it will automatically detects type of vehicles using during the trip. If the trip organized as a team, the users will be able to share their in-team location via the Internet so that team members can follow each other. It is intended that a route created and shared by a user or a team can be examined by other users. Other users may choose the route they are reviewing as their route and they can start trips on this route as personal or team trip. Mobile application will show chosen path on the map different from user's. The system also need to create a social media environment to increase interaction between users. Friendship and tracking system between users is designed and also personalized news flow which includes shared trips by user's friends and popular trips, is provided. As a result of the project, a mobile application that stores route data of a trip and media like photos, videos, audio files tagged on the route is developed, also provides service to interact people with shared data. Sharing trips is an additional workload for travelers. This application offerx users a practical solution to save the time spent on sharing a trip on any social media environment or on the internet. In addition, the application generate convenience not only for travelers but also for people who want to share their daily life. It also provides an open environment for the interaction of people as it is considered to be a social media environment within the application.
}


\def\abstractKeywordsEnglish{
    % Buraya İngilizce olarak proje için geçerli anahtar kelimeleri yazınız
   Trip, Tracker, Advisor, Social Media, Gallery Editor
}

\def\abstractTextTurkish{
    Bu projenin hedefi insanların gezilerininin mobil cihaz konum verileri kullanılarak gezi güzergâhının kaydedilmesi ve güzergâh üzerinde medya ekleyip düzenlenmesini sağlamaktır. Gezilerin sadece kişiler olarak değil, takım hâlinde de düzenlenebilmesi, ayrıca yapılacak gezilerde varsa kullanılan ulaşım araçlarının neler olduğunu algılayabilecek bir sistem geliştirilmiştir. Takım olarak düzenlenen gezilerde kullanıcılar, takım üyelerinin birbirini takip etmesi amacıyla takım içi konum paylaşımını Internet bağlantısı olması şartıyla yapabilirler. Bir kullanıcının veya takımın oluşturduğu bir güzergâhı, diğer kullanıcılar da inceleyebilmektedir. Diğer kullanıcılar inceleme yaptığı güzerâhı kendi güzergâhı olarak belirleyebilir veya bu güzergâh üzerinde takım gezisi yapmak isteyebilir. Mobil uygulama belirlenen güzergâhı, kullanıcının güzergâhından farklı olarak gösterecektir. Sistemde kullanıcılar arasında etkileşimi artırmak amacıyla bir sosyal medya ortamı oluşturulmuştur. Kullanıcılar arası arkadaşlık ve takip sistemi tasarlanmıştır. Ayrıca kişiye kendi çevresinden tavsiye edilender, uygulama içinde bulunan popüler güzergâhlar ve arkadaş çevresindeki geziler derlenerek özel bir haber akışı içerisinde sunulmaktadır.
	Proje sonucunda bir gezinin rotasını, rota üzerine etiketlenen fotoğraflarını, videolarını, ses dosyalarını saklamaya imkan veren ve bunları diğer kullanıcıların etkileşimine açabilen bir mobil uygulama geliştirilmiştir. Gezilerin herhangi bir sosyal medya ortamında veya Internet üzerinde paylaşılması ve paylaşmak amacıyla düzenlenmesi geziciler için ek iş yükü teşkil etmektedir. Bu uygulama, kullanıcılarına gezi sürecinin paylaşımında sarf edilen emek ve zamandan tasarruf ettirecek pratik bir çözüm sunmaktadır. Ayrıca uygulama sadece gezicileri değil günlük hayatını paylaşmak isteyen insanlar içinde kolaylık üretmarktedir. Uygulama içinde bir sosyal medya ortamının da olması düşünüldüğü için kişilerin etkileşimine açık bir ortam sağlanmıştır.

}

\def\abstractKeywordsTurkish{
     Gezi, Takip, Öneri Sistemi, Sosyal Medya, Galeri Düzenleyici
    
}

% Proje için gerekli olan sistem ve yazılım bilgilerini yazınız.
\def\software{Linux, Java, Android Studio, MySQL, Android 5.0 or higher, Spring Boot,Javascript, AngularJS, Bootstrap, CSS, Android Emulator, Postman, JUnit, Google Map API, Mail Service, Android Phone,Git,Eclipse }

% Proje için gerekli olan RAM bellek boyutunu yazınız.
\def\memorysize{Mobile Application: 1.5GB\\ Web Application: 8GB}

% Proje için gerekli olan harddisk boyutunu yazınız.
\def\disksize{Mobile Application: 8GB \\ Web Application: 1TB}