%%%%%%%%%%%%%%%%%%%%%%%%%%%%%%%%%%%%%%%%%%%%%%%%%%%%%%%%%%%%%%%%%%%%%%%
%%%%%%%%% Aşağıda istenilen bilgileri dikkatlice doldurunuz.   %%%%%%%%
%%%%%%%%% Doldurmanız istenilen ifadenin sonunda TR ya da EN   %%%%%%%%
%%%%%%%%% yazıyorsa, sırasıyla Türkçe veya İngilizce olarak    %%%%%%%%
%%%%%%%%% doldurunuz. Eğer herhangi bir ifade yoksa, projenizi %%%%%%%%
%%%%%%%%% hangi dilde yazıyorsanız (Türkçe veya İngilizce), o  %%%%%%%%
%%%%%%%%% dile göre doldurunuz. İsimleri yazarken soyisimleri  %%%%%%%%
%%%%%%%%% büyük harf ile yazınız.                              %%%%%%%%
%%%%%%%%%%%%%%%%%%%%%%%%%%%%%%%%%%%%%%%%%%%%%%%%%%%%%%%%%%%%%%%%%%%%%%%
%%%%%%%%%%%%%%%%%%%%%%%%%%%%%%%%%%%%%%%%%%%%%%%%%%%%%%%%%%%%%%%%%%%%%%%


% Proje başlığını Türkçe olarak yazınız.
\def\titleTR{ }

% Proje başlığını İngilizce olarak yazınız.
\def\titleEN{ Gezi-Yorum }

% Proje grubundaki ilk ismi yazınız.
\def\studenti{Tarık Nural}
%İlk öğrencinin, öğrenci numarasını yazınız.
\def\numberi{13011036}
% Proje grubundaki ilk öğrencinin doğum tarihi ve yerini yazınız.
\def\studentibdate{07.01.1995, İstanbul}
% Proje grubundaki ilk öğrencinin e-mail adresini yazınız.
\def\studentiemail{tariknural00@gmail.com}
% Proje grubundaki ilk öğrencinin cep telefonu numarasını yazınız.
\def\studentiphone{ 0535 261 1429}
% Proje grubundaki ilk öğrencinin staj deneyimlerini yazınız. Satır atlatmak için \\ kullanabilirsiniz.
\def\studentiintern{ SSI SCHAEFER \\ ARÇELİK A.Ş}

% Proje grubundaki ikinci ismi yazınız. Eğer ikinci üye yoksa ~ işareti ekleyiniz ve ikinci
% öğrenci ile alakalı diğer bilgileri atlayınız.
\def\studentii{Murat Baki Yücel}
%İkinci öğrencinin, öğrenci numarasını yazınız.
\def\numberii{13011035}
% Proje grubundaki ikinci öğrencinin doğum tarihi ve yerini yazınız.
\def\studentiibdate{04.01.1996, Kayseri}
% Proje grubundaki ikinci öğrencinin e-mail adresini yazınız.
\def\studentiiemail{bakiyucel38@gmail.com}
% Proje grubundaki ikinci öğrencinin cep telefonu numarasını yazınız.
\def\studentiiphone{ 0507 915 8686}
% Proje grubundaki ikinci öğrencinin staj deneyimlerini yazınız. Satır atlatmak için \\ kullanabilirsiniz.
\def\studentiiintern{ Pronic Yazılım \\ Kartaca Bilişim}

% Projeyi teslim ettiğiniz ay ve yılı proje için kullandığınız dilde yazınız.
\def\date{Kasım, 2017}

% Proje danışmanınızın ismini Türkçe ünvanı ile yazınız.
\def\advisorTR{Yrd. Doç. Dr. Ahmet Tevfik İNAN}
% Proje danışmanınızın ismini İngilizce ünvanı ile yazınız.
\def\advisorEN{Assist. Prof. Dr. Ahmet Tevfik İNAN}

\def\acknowledgementText{
    % Buraya teşekkür metninizi proje için kullandığınız dilde yazınız. 
    In the realization of this study, our precious faculty members who shared their
    knowledge with us for four years, our learning process does not spare our help and
    they continuously transmit their experiences to us, We would like to thank you for your
    support to the "Assist. Prof. Dr. Ahmet Tevfik İNAN" who has constantly interacted
    with us in the project process and found our way.
}

\def\abstractTextEnglish{
    % Buraya İngilizce olarak proje özetini yazınız.
The goal of this project is to record travel routes of people's trips using mobile device location and to add and edit media on the route. A system will be developed that can be used to organize trips not only as people but also as a team, as well as what transport vehicles are used if they are to be visited. In the tour organized as a team, the users will be able to share their in-team location via the internet in order for team members to follow each other. It is intended that a route created by a user or a team can be examined by other users. Other users may choose the route they are reviewing as their route, or they may want to take a team tour on this route. The system to be designed will be guiding in this case. The system will need to create a social media environment to increase interaction between users. Friendship and tracking system between users will be designed. In addition personalized news flow will be provided. The person will be provided with a customized news flow that will be compiled around the person, compiled on popular routes and in the circle around friends.
As a result of the project, a mobile application will be developed that stores route of a trip and medias like photos, videos, audio files tagged on the route also will provide service to interact people with shared data. Sharing trips is an additional workload for travelers. This application will offer users a practical solution to save the effort spent time on sharing a trip. In addition, the application will generate convenience not only for travelers but also for people who want to share their daily life. It will also provide an open environment for the interaction of people as it is considered to be a social media environment within the application.
}


\def\abstractKeywordsEnglish{
    % Buraya İngilizce olarak proje için geçerli anahtar kelimeleri yazınız
   Trip, Tracker, Advisor, Social Media, Gallery Editor
}

\def\abstractTextTurkish{
    Bu projenin hedefi insanların gezilerininin mobil cihaz konum verileri kullanılarak gezi güzergâhının kaydedilmesi ve güzergâh üzerinde medya ekleyip düzenlenmesini sağlamaktır. Gezilerin sadece kişiler olarak değil, takım hâlinde de düzenlenebilmesi, ayrıca yapılacak gezilerde varsa kullanılan ulaşım araçlarının neler olduğunu algılayabilecek bir sistem geliştirilecektir. Takım olarak düzenlenen gezilerde kullanıcılar, takım üyelerinin birbirini takip etmesi amacıyla takım içi konum paylaşımını internet aracılığı ile yapabilecektir. Bir kullanıcının veya takımın oluşturduğu bir güzergâhı, diğer kullanıcıların da inceleyebilmesi hedeflenmektedir. Diğer kullanıcılar inceleme yaptığı güzerâhı kendi güzergâhı olarak belirleyebilir veya bu güzergâh üzerinde takım gezisi yapmak isteyebilir. Tasarlanacak sistem bu durumda yol gösterici olacaktır. Sistemin kullanıcılar arasında etkileşimi artırmak amacıyla bir sosyal medya ortamı oluşturması gerekecektir. Kullanıcılar arası arkadaşlık ve takip sistemi tasarlanacaktır. Ayrıca kişiye kendi çevresinden tavsiye edilender, uygulama içinde bulunan popüler güzergâhlar ve arkadaş çevresindeki geziler derlenerek özel bir haber akışı içerisinde sunulacaktır.
	Proje sonucunda bir gezinin rotasını, rota üzerine etiketlenen fotoğraflarını, videolarını, ses dosyalarını saklamaya imkan veren ve bunları diğer kullanıcıların etkileşimine açabilen bir mobil uygulama geliştirilecektir. Gezilerin paylaşılması geziciler için ek iş yükü teşkil etmektedir. Bu uygulama, kullanıcılarına gezi sürecinin paylaşımında sarf edilen efordan tasarruf ettirecek pratik bir çözüm sunacaktır. Ayrıca uygulama sadece gezicileri değil günlük hayatını paylaşmak isteyen insanlar içinde kolaylık üretecektir. Ayrıca uygulama içinde bir sosyal medya ortamının da olması düşünüldüğü için kişilerin etkileşimine açık bir ortam sağlayacaktır.

}

\def\abstractKeywordsTurkish{
     Gezi, Takip, Öneri Sistemi, Sosyal Medya, Galeri Düzenleyici
    
}

% Proje için gerekli olan sistem ve yazılım bilgilerini yazınız.
\def\software{ Windows, Linux, Java, Android Studio, MySQL, Android, Spring Boot,Javascript, AngularJS, Mocha and Chai, Bootstrap, CSS, Android Emulator, Postman, JUnit, Google Map API, Mail Service, Android Phone,Git  }

% Proje için gerekli olan RAM bellek boyutunu yazınız.
\def\memorysize{1GB-512MB}

% Proje için gerekli olan harddisk boyutunu yazınız.
\def\disksize{512MB-2GB}